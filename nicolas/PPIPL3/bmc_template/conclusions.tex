Based on previous work from other authors\cite{Kovacs2019}, the intrinsic
knowledge of existing protein-protein interaction networks let scientists
elucidate promising new interactions to explore. In this study, a
machine learning algorithm was fed with previously elucidated information
about the network, as well as results form a state-of-the-art edge
embedding technique. Even if the baseline of the handcrafted features
(A2, A3, L3) yields good results, boosting that information with the
vector embedding, which calculates vector representations of the network
structure improves the overall performance of each method on its own.

On both human interactomes (\emph{HI-II-14} and \emph{HI-TESTED}),
applying \texttt{XGBoost} with the metrics yield results with high
AUC for the link prediction problem, and therefore this study extends
previous work by other authors that only consider the predictive power
of the metrics alone. In this case, the importance of the handcrafted
features was very significant and stands out of the Node2Vec features.

On the rice interactome, PPI prediction obtained high AUC values when
only using the node embedding technique (AUC=0.93) and when only using
the handcrafted features based on protein neighborhood (AUC \textasciitilde 0.88).
Even higher AUC results were achieved when the handcrafted features
and the vector representation information were combined (AUC \textasciitilde 0.98).

Importance values for the A3 method stand out less from the node embedding
information than in A2 and L3. However, handcrafted features are more
important for the prediction of the human than in the rice interactome. 

With this study, a framework for link prediction on protein-protein
interactions is presented and validated against the available information
on rice and human interactomes. However, it might be easily extended
to other organisms or even to other graphs not necessarily related
to biology, because the only required information is the network connectivity.
Satisfactory results might be expected when the framework is applied
to networks with a dense adjacency matrix, since calculations of L3
and Node2Vec rely on network exploration for an arbitrary node and
its degree.
