\parttitle{Background} %if any
Abstract A recent study in network-based prediction of protein-protein interactions (PPIs) reveals that two proteins are more likely to interact, the higher the number of paths of length 3 between them (normalized by the geometric average of their interactions). This paper extends previous work on mapping binary interactions by taking into account the learning of features (embeddings) of the PPI network. In particular, we implement a gradient boosted decision tree model (XGBoost) using handcrafted features (including the normalized measure) and embeddings from an algorithm that generates a low-dimensional representation of nodes (node2vec).

\parttitle{Results} %if any
Our main result shows that while the measure remains an important feature for predicting interactions, better performance is achieved when in addition embedding features are considered. The proposed approach is validated for the human and rice interactomes. For both cases, the combination of both types of features yield higher AUC values.

\parttitle{Conclusions} %if any
As found on this study on both human and rice, when information from handcrafted features based on neighborhood is enhanced with vector representations from random walks, the prediction power of the model improves. Besides, a supervised learning model can be trained for predicting unknown interactions based on such information. Finally, the developed framework can also be applied to interactomes of other organisms for which PPI networks have recently become available.