\parttitle{Background}
Recent research on predicting protein-protein
interactions (PPIs) indicates that two proteins are more likely to interact if 
there exists a higher number of paths of length 3 between them. This paper
extends network-based approaches to map binary interactions between proteins 
by combining hand-crafted features such as the number of paths between proteins 
and learned features (i.e. node embeddings) of the PPI network. In particular, 
the proposed approach introduces a gradient boosted decision tree model 
(XGBoost) usese node2vec to infer embeddings, that is, a low-dimensional representation
of the proteins and their interactions.

\parttitle{Results}
The main result shows that while the normalized measure remains an important feature
for predicting interactions, embeddings may improve performance of predictions by up 
to 15\% in terms of the Area Under the Curve metric (i.e. AUC). The approach is validated for 
the human and rice interactomes.

\parttitle{Conclusions}
As shown by the two case studies in human and rice, combining information from 
handcrafted features with embeddings enhances the prediction power of the model. 
Improvements of supervised learning models (such as XGBoost) depend on being able 
to extract additional information about the local neighborhood of nodes that 
complement the measure based on the number of paths. Finally, the proposed 
approach could also be applied to interactomes of a number of other organisms for which
PPI networks have recently become available.