\parttitle{Background}
Recent research on network-based prediction of protein-protein
interactions (PPIs) reveals that two proteins are more likely to interact,
the higher the number of paths of length 3 between them exist. This paper
extends previous work on mapping binary interactions between proteins 
by learning features (i.e. embeddings) of the PPI network. In particular, 
this paper reports on an application of gradient boosted decision tree
model (XGBoost) using handcrafted features  such as normalized measures 
and algorithmic embeddings that generate a low-dimensional representation
of the proteins and their interactions (node2vec).

\parttitle{Results}
The main result shows that while the measure remains an important feature
for predicting interactions, an improvement in performance is achieved when
embedding features are considered. The proposed approach is validated for
the human and rice interactomes. For both cases, the combination of both
types of features yield higher performance in terms of the Area Under the
Curve metric (i.e. AUC).

\parttitle{Conclusions}
As witnessed by both human and rice, when information from 
handcrafted features based on neighborhood is enhanced with vector
representations from random walks, the prediction power of the model
improves. Moreover, a supervised learning model has been trained to predict
unknown interactions based on such information. Finally, the developed
framework could also be applied to interactomes of other organisms for which
PPI networks have recently become available.