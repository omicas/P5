\section*{Concluding Remarks}
\label{sec.concl}

This manuscript provides a detailed description of a network-based
analysis workflow for the discovery of key genes responding to a
specific treatment in an organism. It links transcriptomic with
phenotypic data and identifies overlapping gene modules.
\vspace{0.5cm}

The proposed approach is inspired by the workflow suggested in the
WGCNA~\cite{langfelder2008wgcna}. Its main steps are the preprocessing
of the gene expression data, the construction of a co-expression
network, the detection of modules within the network, the relation of
modules with external information (e.g., phenotypic data), and the
enrichment of the identified key genes with additional information.
Both approaches are structured in a modular way, which allows
modifying and exploring different techniques in each step of the
workflow.
\vspace{0.5cm}

The proposed workflow is designed to integrate expression data
measured under two different conditions (namely, control and
treatment), unlike the usually co-expression-based approaches which
work with both conditions independently or consider only a single
condition. For this purpose, an approach similar to that proposed
in~\cite{du2019network} is used, where the control and treatment data
are compiled in a single matrix using the Log Fold Change
measure. Thus, the input to construct the co-expression network is not
the expression data, but instead the changes in the expression levels
from one condition to the other, making room for capturing the signal
of changes caused by the treatment.
\vspace{0.5cm}

An important feature in the proposed workflow is the module
detection technique. The co-expression network is computed, as in
WGCNA, until a scale-free network is obtained. In the proposed
approach, this network is then used to apply the HLC algorithm, a
clustering technique capable of detecting overlapping
communities. Several approaches of module detection from gene
expression have been proposed and were evaluated
in~\cite{saelens2018comprehensive}. Most of them focus mainly on
disjoint (non-overlapping) communities; the techniques described
dealing with overlaps are not clustering, but bi-clustering and
decomposition methods. It is well known that communities in real
networks, including biological ones, are likely
overlap~\cite{palla2005uncovering}. Thus, the approach presented
in this work can be seen as a generalization of the previous
approaches, such as WGCNA, with the potential to deal with genes
associated to multiple biological processes.
\vspace{0.5cm}

The approach was applied in a case study with rice under salt
stress. The results show a group of 14 genes, of which only $2$ of
them have been previously related to saline stress response in other
studies. As future work, other overlapping module detection and
selection techniques should be used instead HLC and LASSO,
respectively. The combination of these techniques would allow finding
target genes for future biological studies that evaluate their
potential as genes that respond to salt stress in rice, and other
crops and stresses. In-vivo laboratory experimentation needs to be
conducted to validate the findings of this paper in relation to
salinity stress.
\vspace{0.5cm}

Finally, the workflow is presented as a protocol capable of
considerably reducing the number of genes detected as relevant in the
response to stress of choice. Other traditionally used methods for
this purpose tend to generate a large list of candidate genes, thus
limiting subsequent efforts in experimental validation. In this sense,
the proposed workflow can help in reducing such efforts in time and
money invested by researchers in the experimental validation of
stress-responsive genes.
