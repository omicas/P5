\begin{abstract}
  This paper proposes a workflow to identify genes responding to a
  specific treatment in an organism, such as abiotic stresses, a main
  cause of extensive agricultural production losses worldwide.  On
  input RNA sequencing read counts (measured for genotypes under
  control and treatment conditions) and biological replicates, it
  outputs a collection of characterized genes, potentially relevant to
  treatment. Technically, the proposed approach is both a
  generalization and an extension of WGCNA; its main goal is to
  identify specific modules in a network of genes after a sequence of
  normalization and filtering steps. In this work, module detection is
  achieved by using Hierarchical Link Clustering, which can recognize
  overlapping communities and thus have more biological meaning given
  the overlapping regulatory domains of systems that generate
  co-expression. Additional steps and information are also added to
  the workflow, where some networks in the intermediate steps are
  forced to be scale-free and LASSO regression is employed to select
  the most significant modules of phenotypical responses to stress.
  %
  %% Biologists may use these genes to develop new crop cultivars with
  %% higher resistance to a specific treatment.
  %
  Finally, the workflow is showcased with a systematic study on rice
  (\textit{Oryza sativa}), a major food source that is known to be
  highly sensitive to salt stress: a total of 6 modules are detected
  as relevant in the response to salt stress in rice;
  %
  %% : 3 modules of 3 genes each, all associated with shoot K content,
  %% 2 modules of 3 genes associated with shoot biomass, and 1 module
  %% of 4 genes associated with root biomass.
  %
  these genes may act as potential targets for the improvement of
  salinity tolerance in rice cultivars.
  %
  The proposed workflow has the potential to ultimately reduce the
  search-space for candidate genes responding to a specific treatment,
  which can considerably optimize the effort, time, and money invested
  by researchers in the experimental validation of stress responsive
  genes.
  \keywords{stress responsive genes \and co-expression network 
            \and overlapping communities \and phenotypic traits
            \and LASSO regression.}
\end{abstract}
