\section{Introduction}
\label{sec.intro}

Abiotic stresses are key factors that can negatively influence plant
development and productivity. They are a main cause of extensive
agricultural production losses worldwide~\cite{mesterhazy2020losses}. Soil salinity is
one of the most devastating abiotic stresses, causing reduction in the
cultivable land, crop quality, and productivity. It has been estimated
that 20\% of total cultivated and 33\% of irrigated agricultural lands
worldwide are already affected by high salinity. Moreover, due to the
human activities and natural causes, salinized areas are gradually
increasing every year and are expected to reach 50\% by the end of
year 2050~\cite{shrivastava2015soil}. Salinity tolerance and
susceptibility in plants is known to be the result of elaborated
interactions between morphological, physiological, and biochemical
processes that are regulated, in the end, by multiple genes in
different parts of a genome~\cite{reddy2017salt}. Therefore,
identifying groups of stress responsive genes may lead to crop
improvement in terms of salinity tolerance and, ultimately, contribute
solutions to the general problem of food sustainability in the years
to come.


This paper proposes a workflow to identify stress responsive genes in
organisms, which is known to be a complex quantitative trait. It takes
as input RNA sequencing read counts measured for genotypes under
control and treatment conditions (and representing gene expression
profiles of the target organism), and biological replicates.  In order
to discover key genes and their interaction with phenotypes related to
treatment tolerance, the approach requires a collection of phenotypic
traits (under control and under treatment), measured for the given
genotypes. The output of the workflow is a collection of characterized
genes, potentially relevant to treatment, yielding insight on the
possible behavior of specific genes and the role they may play in
functional pathways in response to the studied treatment of the
organism of interest. The proposed workflow can thus take advantage of
transcriptomic data for different organisms and conditions, based on
the current availability of high-throughput technologies that include
microarrays and RNA sequencing, to study the reaction of organisms
under different environmental stimuli, such as salt stress.


Technically, the proposed approach is both a generalization and an
extension of Weighted Gene Co-expression Network Analysis
(WGCNA)~\cite{langfelder2008wgcna}, a widely applied workflow that has been
successfully used for identifying target genes related to diseases and
cancer in several organisms~\cite{tian2018identifying}. The general
idea behind each approach is to identify specific modules in a network
of genes after a sequence of normalization and filtering steps. The
proposed approach is considered a \textit{generalization} of WGCNA
because module detection can now recognize overlapping communities,
which may have more biological meaning given the overlapping
regulatory domains of systems that generate
co-expression~\cite{gaiteri2014beyond}. This is achieved by using
Hierarchical Link Clustering (HLC)~\cite{ahn2010link}. It is also an
\textit{extension} of WGCNA because additional steps and information
are added to the workflow: namely, some networks in the intermediate
steps are forced to be scale-free~\cite{todo3} and LASSO
regression~\cite{tibshirani1996regression} is employed to select the
most significant modules of phenotypical responses to stress.  The
advantage of using HLC as clustering method is its ability to detect
overlapping modules, since biological components are involved in
multiple functions and therefore biological communities tend to be
highly overlapping. On the other hand, LASSO is a regularized
regression technique widely used in variable selection, thanks to its
ability to obtain zero regression coefficients for the less relevant
variables~\cite{desboulets2018review}. Moreover, LASSO is especially
useful in problems where the number of variables is much larger than
the number of samples, which may be the case more often than desired.
The proposed workflow is also modular, since other module detection
and selection techniques could be used, instead HLC and LASSO,
respectively.


The proposed workflow is showcased with a systematic study on rice
(\textit{Oryza sativa}), a major food source that is known to be
highly sensitive to salt stress~\cite{chang2019morphological}. RNA-seq
data was accessed from the GEO database~\cite{GEOAcces90:online}
(accession number GSE98455). It corresponds to $57845$ gene expression
profiles of shoot tissues measured for both control and salt condition
in $92$ accessions of the Rice Diversity Panel 1. As output, 6 modules
are detected as relevant in the response to salt stress in rice: 3
modules of 3 genes each, all associated with shoot K content, 2
modules of 3 genes associated with shoot biomass, and 1 module of 4
genes associated with root biomass. These genes may act as potential
targets for the improvement of salinity tolerance in rice
cultivars. From the 19 genes, all but 3 genes (associated with $K$
content), were also identified as deferentially expressed for at least
one of the 92 accessions, suggesting that those genes are strong
candidates as stress responsive genes. Only 2 of the 16 diferentially
expressed genes, both from the module related with shoot biomass, are
named and have an associated protein product: Spermidine
hydroxycinnamoyltransferase 2 (SHT2) and Lipoxygenase. In other words,
further studies are needed to elucidate the detailed biological
function of the remaining 14 genes that have not been named so far,
which may have a potential relevance in stress responsive mechanisms
to salt conditions in rice. The goal is that the results reported in
this paper may allow biologist to develop new rice cultivars with
higher resistance to salinity.
