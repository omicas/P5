\section*{Introduction}
\label{sec.intro}

Stresses are key factors that influence plant
development, often associated to extensive
losses in agricultural production~\cite{mesterhazy2020losses, shrivastava2015soil}. 
Soil salinity is one of the
most devastating abiotic stresses. According to~\cite{shrivastava2015soil}, 
soil salinity contributes to a significant reduction in areas of cultivable
land and crop quality. The study estimates that 
20\% of the total cultivated land worldwide and 
33\% of the total irrigated agricultural land
is affected by high salinity. 
By the end of 2050 areas of high salinity are
expected to reach 50\% of the cultivated land~\cite{shrivastava2015soil}. 
\vspace{0.5cm}

Salinity tolerance and
susceptibility are the result of elaborated
interactions between morphological, physiological, and biochemical
processes, which are regulated by multiple genes in
various parts of the plant genome~\cite{reddy2017salt}. Consequently,
identifying groups of responsive genes is an important step in efforts to
improve crop varieties in terms of salinity tolerance.
This paper proposes a workflow to identify stress responsive genes
associated with a complex quantitative trait.
\vspace{0.5cm}

To discover which genes are associated with a phenotypic response to
treatment the workflow takes as input
the gene expression profiles of the target organism, specifically, 
the RNA sequencing read counts (measured under control
and treatment conditions)
of at least two biological replicates per genotype. It also receives phenotypic data, specifically, observable traits, measured for each genotype under the two conditions. The output of the
workflow is a set of genes which are characterized as potentially relevant
to treatment. 
\vspace{0.5cm}

Broadly speaking, the workflow provides a framework that yields insight into the possible behavior of specific
genes and the role they play in functional pathways in response to
treatment. 
It takes advantage of
the current availability of high-throughput technologies, which
enables the access to transcriptomic data of organisms under different conditions, 
and a better understanding of their reaction under different
environmental stimuli.
\vspace{0.5cm}

The proposed approach is both a generalization and an
extension of the widely applied workflow 
for identifying target genes called 
Weighted Gene Co-expression Network Analysis
(WGCNA)~\cite{langfelder2008wgcna, tian2018identifying}.
Like WGCNA, the general idea behind the proposed
approach is to identify, after a sequence of normalization
and filtering steps, specific modules of overlapping communities
underlying the co-expression network of genes. 
The proposed approach is considered a \textit{generalization}
of WGCNA because module detection recognizes overlapping communities
using Hierarchical Link Clustering (HLC)~\cite{ahn2010link} algorithm.
Conceptually, the generalization takes into account
the overlapping nature of the regulatory domains of the
systems that generate the co-expression network~\cite{gaiteri2014beyond}.
More specifically, overlapping modules allow for 
scenarios where biological components  are involved in
multiple functions.
\vspace{0.5cm}

The workflow is also an \textit{extension} of WGCNA
because two additional constraints are considered:
networks in the intermediate steps are forced to be
scale-free~\cite{barabasi2003scale}, and LASSO
regression~\cite{tibshirani1996regression} selects the
most relevant modules of responsive genes.
The regularized regression technique of LASSO
forces the less relevant variables to be associated to regression
coefficients of value zero~\cite{desboulets2018review}, 
which is of interest
in scenarios where the number of variables is much larger than
the number of samples. This condition is satisfied when the target variables represent the
overlapping communities (obtained with HLC) and the samples represent genotype data,
which is usually a small set due to the high cost of the RNA sequencing process. 
Finally, the proposed workflow is also modular, since other module detection
and selection techniques could be explored instead HLC and LASSO.
\vspace{0.5cm}

The approach is showcased with a systematic study on rice
(\textit{Oryza sativa}), a food source that is known to be
highly sensitive to salt stress~\cite{chang2019morphological}. RNA-seq
data was accessed from the GEO database~\cite{clough2016gene}
(accession number GSE98455). It represents $57845$ gene expression
profiles of shoot tissues measured under control and stress conditions
in $92$ accessions of the Rice Diversity Panel 1~\cite{eizenga2014registration}. A total of 6 modules
are detected as relevant in the response to salt stress in rice: 3
modules, each grouping together 3 genes, are associated to shoot K content; 2
modules of 3 genes, are associated to shoot biomass; and 1 module of 4
genes is associated to root biomass. These genes are potential
targets for experimental validation of salinity tolerance. 
From the 19 genes, 16
are also identified as deferentially expressed for at least
one of the 92 accessions, which re-enforces the labelling of the genes as
stress responsive. Only 2 of the 16 differentially
expressed genes, associated to shoot biomass, are
named and known to produce protein products: spermidine
hydroxycinnamoyltransferase~2 (SHT2) and lipoxygenase. 
Further studies are needed to elucidate the detailed biological
functions of the remaining 14 genes and their role
in the mechanisms that respond 
to salt conditions.

\paragraph{Paper Outline.}
%The remainder of the paper is organized as
%follows. Section~\ref{sec.prelim} gathers preliminaries on gene
%co-expression networks, HLC, and LASSO. The proposed workflow is
%presented in Section~\ref{sec.framework}, with especial focus on the
%logical steps of the process and the internal structures supporting
%the approach. Section~\ref{sec.case} presents the case study on the
%identification of rice genes that are sensitive to salt
%stress. Section~\ref{sec.concl} draws some conclusions and future research directions.

The remainder of the paper is organized as
follows. The \hyperref[sec.prelim]{Preliminaries} section gathers foundations on gene
co-expression networks, HLC, and LASSO. The proposed workflow is
presented in \hyperref[sec.framework]{the Workflow} section, which enphasizes on the
logical steps of the data analysis process and the internal structures supporting
the approach. The \hyperref[sec.case]{Case Study} section presents an application of the workflow for the
identification of rice genes that are sensitive to salt
stress. Finally, the \hyperref[sec.concl]{Concluding Remarks} section draws some conclusions and future research directions.