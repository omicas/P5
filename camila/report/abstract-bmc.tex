\begin{abstract} % abstract
\parttitle{Background} %if any
This paper proposes a workflow to identify which genes respond to
  specific treatments in plants.
  The workflow takes as
  input the RNA sequencing read counts and phenotypical data of different genotypes,
  measured under  control and treatment conditions.
  It outputs a reduced group of genes marked as relevant for
  treatment response. Technically, the proposed approach is both a
  generalization and an extension of 
%  Weighted Gene Co-expression Network Analysis (WGCNA).
  WGCNA.
  It aims to 
  identify specific modules of overlapping communities underlying the co-expression network of genes.
  Module detection is
  achieved by using Hierarchical Link Clustering.
  Identifying such modules enables us to take into account
  the overlapping nature of the regulatory domains of the systems that generate
  co-expression. LASSO regression is employed to analyze
  phenotypic responses of modules to treatment.

\parttitle{Results} %if any
  The workflow is applied to rice
  (\textit{Oryza sativa}), a major food source known to be
  highly sensitive to salt stress.
  The workflow identifies 19 rice genes that seem
  relevant in the response to salt stress.
  They are
  distributed across 6 modules: 
  3 modules, each grouping together 3 genes, are associated to shoot K content;
  2 modules of 3 genes, are associated to shoot biomass;
  and 1 module of 4 genes is associated to root biomass.
  These genes represent target genes for the improvement of
  salinity tolerance in rice.
   
  
\parttitle{Conclusions}
A more effective framework to reduce the
  search-space for target genes that respond to a specific treatment is introduced.
  It facilitates experimental validation by restraining efforts to
  a smaller subset of genes of high potential relevance.
  
\end{abstract}